Speaker: Eric C. Anderson

Affiliation: Southwest Fisheries Science Center, NOAA Fisheries, Santa Cruz, CA (academic visitor at Oxford University Dept. of Zoology)

Title:  "Reaching haplotopia: the promise and practice of short-read microhaplotypes in molecular ecology"

Abstract: 

Next-generation sequencing (NGS) technologies have revolutionized data collection in molecular ecology, providing many new insights.  Most of these new insights have come from sequencing a relatively small number of individuals at thousands to millions of sites in the genome. Less familiar, though no less revolutionary, are applications of NGS to the economical genotyping of tens of thousands of individuals at a relatively small number (100s) of SNPs.  In the first half of the talk I will motivate this latter topic by reviewing applications of SNP genotyping in the conservation and management of Pacific salmon, and will describe the role that NGS technologies now play.  In the second half, I will describe my lab's ongoing bioinformatic and statistical developments that permit the computationally efficient treatment of such NGS data as multiallelic markers that we call "microhaplotypes."  Like microsatellites, these markers can provide substantially more power for relationship inference than individual SNPs, and they are easily portable between species.  However, unlike microsatellites, microhaplotypes are easily standardized between laboratories and are far less subject to homoplasy.  As such, microhaplotypes are a promising new marker for many applications in conservation genetics, including pedigree reconstruction, assessment of population structure, kinship estimation, design of captive breeding, and more.  I illustrate the utility of microhaplotypes in the context of identifying larval self-recruitment in a coastal California rockfish species.   

